\documentclass[10pt,a4paper]{article}
\usepackage[utf8]{inputenc}
\usepackage[czech]{babel}
\usepackage[T1]{fontenc}
\usepackage{amsmath}
\usepackage{amsfonts}
\usepackage{amssymb}
\usepackage[left=2cm,right=2cm,top=2.5cm,bottom=2cm]{geometry}
\author{Stanislav Nechutný}
\begin{document}
Dokumentace úlohy JSN: JSON2XML v~PHP 5 do IPP 2014/2015

Jméno a příjmení: Stanislav Nechutný

Login: xnechu01

\section{Základní návrh}
	Pro implementaci byl zvolen objektový návrh a pro zpracování chyb výjimky, které zajišťují zpřehlednění a zkrácení kódu.
	
	\textbf{Tok programu je navržen:} Načtení argumentů $\rightarrow$ Načtení souboru $\rightarrow$ Načtení JSONu $\rightarrow$ Vytvoření struktury objektů pro vnitřní reprezentaci dat $\rightarrow$ Serializace do XML $\rightarrow$ Zápis do souboru
	
	Celý tok programu může být v~jakémkoliv stádiu provádění přerušen vyvoláním výjimky. V~takovém případě je výjimka odchycena, zobrazena příslušná chybová zpráva na standardní chybový výstup a program ukončen s~odpovídajícím návratovým kódem.
	
\section{Zpracování argumentů}
	Pro zpracování argumentů spouštěného programu byla vytvořena statická třída \textit{arguments}, která se stará o~jejich načtení, validaci a zpřístupnění zbytku programu při jeho běhu.
	
	Metoda \textit{load} iteruje nad argumenty programu, předanými jako pole, a provádí jejich kontrolu. V~případě opakované deklarace, nevalidní hodnoty, či nedovolené kombinace vyhodí výjimku.
	
	Načtené hodnoty jsou po dobu vykonávání programu dostupné přes tzv. gettery, které zohledňují, zda byla hodnota nastavena, nebo zda se má použít výchozí hodnota.
	
\section{Práce se souborem}
	Pro práci se soubory (čtení a zápis) je vytvořena statická třída \textit{file} s~metodami \textit{read} a \textit{write}, které provádí zápis/čtení do souboru. V~případě problémů s~prováděním operací dojde k~vyhození výjimky.
	
\section{JSON}
	Třída implementující funkci pro načtení řetězce do struktury. Pro načtení dat je použita funkce \textit{json\_decode}. Problémem je, že funkce \textit{json\_decode} pro JSON řetězec \textit{"null"}\ vrací správně načtenou hodnotu, která je stejná s~chybovou hodnotou. Z~toho důvodu je zde přidána kontrola na tuto hodnotu.
	
\section{Vnitřní struktura reprezentace dat}
	Pro reprezentaci dat je vytvořena třída \textit{XMLElement}, která ve svém konstruktoru přijímá jméno hodnoty, smíšená data (pole, objekty, literály atd.) a volitelně úroveň odsazení. Konstruktor provede nastavení vlastností objektu a v~případě objektů a polí provede rekurzivně vytvoření dalších podřazených instancí.
	
\section{Serializace do XML}
	Pro serializaci dat implementuje třída \textit{XMLElement} metodu \textit{output}, která vrací řetězec obsahující XML řetězec reprezentující všechny vnořené elementy. Implementace takové metody je řešena rekurzivním voláním stejné metody na podčlenech a konkatenace výsledků.
	
	Při generování serializovaných dat je použita metoda \textit{escapeTagName}, \textit{escapeValue} a \textit{indent}, které oddělují logické části a zajištují generování validního a přehledného výstupu.
	
	Ošetření názvu XML elementu je prováděno pomocí regulérního výrazu, který provede nahrazení nepovolených znaků na daných místech a následně zkontroluje validitu. V~případě nevalidního výsledného souboru dojde k~vyvolání výjimky.
	
	Hodnota a její správné zformátování na výstup je zajištěno metodou \textit{escapeValue}, která na základě argumentů provede nahrazení problematických znaků, zaokrouhlení desetinných čísel, či nahrazení literálů za jejich textovou reprezentaci.
	
	Pro implementaci serializace nemohla být použita metoda \textit{\_\_toString}, jelikož v~této metodě není dovoleno vyvolání výjimky a nebylo by tedy možné vhodně zajistit všechna předepsaná chování.
	
\section{Rozšíření}
	Pro projekt byla implementována veškerá rozšíření.

	\subsection{JPD (0,5b) - Doplnění nul}
		Pro implementaci tohoto rozšíření bylo v~třídě \textit{XMLElement} přidáno doplnění nul zleva za použití funkce \textit{str\_pad}. Výpočet minimálního nutného počtu nul je proveden funkcí \textit{strlen}, které je předáno číslo získané z~součtu počátečního indexu pole s~počtem prvků v~poli - 1.
		
	\subsection{FLA (1.0b) - Zploštění}
		Pro implementaci tohoto rozšíření byla vytvořen třída \textit{flat}, která obsahuje kromě konstruktoru pro načtení dat a metody \textit{output}, pro získání zploštělého výsledku, také metodu pro rekurzivní průchod a funkci pro porovnání hodnot.
		
		Rekurzivní průchod všemi prvky vstupu je zajištěn metodou \textit{recursiveWalk}, která na základě klíče, hodnoty a již zploštělého výsledku provede příslušné přidání. 
		
		Pro korektní řazení hodnot, kdy se řadí literály, čísla i řetězce dle zadané priority je implementována metoda \textit{cmp}, která je navržena, aby ji bylo možné použít ve vestavěné funkci \textit{usort}.
		
	\subsection{JER (1.0b) - Oprava chyb JSON}
		Pro opravu chyb ve vstupním JSONu byla metoda \textit{decode} třídy \textit{json} rozšířena o~parametr fix, který v~případě pravdivé hodnoty zajistí spuštění opravy.
		Opravované chyby:
			
			- Index není umístěn v~uvozovkách.
			
			- Chybějící uvozovka na konci řádku.
			
			- Přebývající čárka před znaky ], nebo \}.
			
			- Chybějící čárka mezi dvěmi řetězci v~poli.
			
			- Chybějící uvozovka za dvojtečkou v~případě řetězce.
			
			- Doplnění chybějících ] a \} na konci JSON řetězce.

\end{document}